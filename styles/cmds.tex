\def\exampletext{Example} % If English

\NewDocumentEnvironment{testexample}{ O{} }
{
\colorlet{colexam}{red!55!black} % Global example color
\newtcolorbox[use counter=testexample]{testexamplebox}{%
    % Example Frame Start
    empty,% Empty previously set parameters
    title={\exampletext: #1},% use \thetcbcounter to access the testexample counter text
    % Attaching a box requires an overlay
    attach boxed title to top left,
       % Ensures proper line breaking in longer titles
       minipage boxed title,
    % (boxed title style requires an overlay)
    boxed title style={empty,size=minimal,toprule=0pt,top=4pt,left=3mm,overlay={}},
    coltitle=colexam,fonttitle=\bfseries,
    before=\par\medskip\noindent,parbox=false,boxsep=0pt,left=3mm,right=0mm,top=2pt,breakable,pad at break=0mm,
       before upper=\csname @totalleftmargin\endcsname0pt, % Use instead of parbox=true. This ensures parskip is inherited by box.
    % Handles box when it exists on one page only
    overlay unbroken={\draw[colexam,line width=.5pt] ([xshift=-0pt]title.north west) -- ([xshift=-0pt]frame.south west); },
    % Handles multipage box: first page
    overlay first={\draw[colexam,line width=.5pt] ([xshift=-0pt]title.north west) -- ([xshift=-0pt]frame.south west); },
    % Handles multipage box: middle page
    overlay middle={\draw[colexam,line width=.5pt] ([xshift=-0pt]frame.north west) -- ([xshift=-0pt]frame.south west); },
    % Handles multipage box: last page
    overlay last={\draw[colexam,line width=.5pt] ([xshift=-0pt]frame.north west) -- ([xshift=-0pt]frame.south west); },%
    }
\begin{testexamplebox}}
{\end{testexamplebox}\endlist}

%% summary
\def\summarytext{Summary} % If English

\NewDocumentEnvironment{testsummary}{ O{} }
{
\colorlet{colexam}{green!55!black} % Global example color
\newtcolorbox{testsummarybox}{%
    % Example Frame Start
    empty,% Empty previously set parameters
    title={\summarytext: #1},% use \thetcbcounter to access the testexample counter text
    % Attaching a box requires an overlay
    attach boxed title to top left,
       % Ensures proper line breaking in longer titles
       minipage boxed title,
    % (boxed title style requires an overlay)
    boxed title style={empty,size=minimal,toprule=0pt,top=4pt,left=3mm,overlay={}},
    coltitle=colexam,fonttitle=\bfseries,
    before=\par\medskip\noindent,parbox=false,boxsep=0pt,left=3mm,right=0mm,top=2pt,breakable,pad at break=0mm,
       before upper=\csname @totalleftmargin\endcsname0pt, % Use instead of parbox=true. This ensures parskip is inherited by box.
    % Handles box when it exists on one page only
    overlay unbroken={\draw[colexam,line width=.5pt] ([xshift=-0pt]title.north west) -- ([xshift=-0pt]frame.south west); },
    % Handles multipage box: first page
    overlay first={\draw[colexam,line width=.5pt] ([xshift=-0pt]title.north west) -- ([xshift=-0pt]frame.south west); },
    % Handles multipage box: middle page
    overlay middle={\draw[colexam,line width=.5pt] ([xshift=-0pt]frame.north west) -- ([xshift=-0pt]frame.south west); },
    % Handles multipage box: last page
    overlay last={\draw[colexam,line width=.5pt] ([xshift=-0pt]frame.north west) -- ([xshift=-0pt]frame.south west); },%
    }
\begin{testsummarybox}}
{\end{testsummarybox}\endlist}

% \newcommand{\sen}{{\rm sen}}
% \newcommand{\Co}{\mathbb{C}}

\makeatletter         
    \def\@maketitle{
    \raggedright
    \begin{center}
        {\Huge \bfseries \sffamily \@title }\\[4ex] 
        {\Large  \@author}\\[4ex] 
        \@date\\[8ex]
    \end{center}}
    \makeatother

\renewcommand{\sfdefault}{phv}
\renewcommand{\familydefault}{\sfdefault}

\newcommand{\ein}{\ensuremath{\in}}
\newcommand{\noin}{\ensuremath{\notin}}

\newcommand{\nset}{\ensuremath{\mathbb{N}\,}}
\newcommand{\zset}{\ensuremath{\mathbb{Z}\,}}
\newcommand{\rset}{\ensuremath{\mathbb{R}\,}}
\newcommand{\qset}{\ensuremath{\mathbb{Q}\,}}
\newcommand{\cset}{\ensuremath{\mathbb{C}\,}}
\newcommand{\serie}[1]{\ensuremath{\mathit{#1}}}
\newcommand{\tb}[1]{\textbf{#1}}
\newcommand{\ti}[1]{\textit{#1}}

\newcommand{\cnum}[3]{\ensuremath{#1\,#2\,#3i}}
\newcommand{\cnuma}[2]{\ensuremath{#1\,+\,#2i}}
\newcommand{\cnums}[2]{\ensuremath{#1\,-\,#2i}}
\newcommand{\ccnum}[1]{\ensuremath{\overline{#1}}}
\newcommand{\conjg}[1]{\ensuremath{#1}$^*$}
\newcommand{\nalpha}{\ensuremath{\alpha}}
\newcommand{\nbeta}{\ensuremath{\beta}}

\newcommand{\npi}{\ensuremath{\pi\,}}
\newcommand{\cmod}[1]{\ensuremath{||#1||}}
\newcommand{\expn}[2]{\ensuremath{$#1^#2$}}

\newcommand{\cmatDD}[2]{\ensuremath{\begin{pmatrix} #1 \\ #2\end{pmatrix}}}
\newcommand{\sqcmatDD}[4]{\ensuremath{\begin{pmatrix} #1 & #2 \\ #3 & #4\end{pmatrix}}}

\newcommand{\dsum}[3]{\displaystyle{ \sum^{#1}_{#2} #3} }

\newcommand{\hlc}[2][yellow]{{%
    \colorlet{foo}{#1}%
    \sethlcolor{foo}\hl{#2}}%
}

%%\rowcolors{2}{gray!15}{white}

\newcommand{\head}[1]{%
\textcolor{white}{\textbf{#1}}}

%%\renewcommand{\arraystretch}{1.5}

%% definir vector
\newcommand{\vt}[1]{\ensuremath{\vec{#1}}}

%% column matrix
\newcommand{\cmdr}[2]{\ensuremath{\begin{pmatrix} #1 \\ #2 \end{pmatrix}}}
\newcommand{\cmtr}[3]{\ensuremath{\begin{pmatrix} #1 \\ #2 \\ #3 \end{pmatrix}}}

%% dot product
\newcommand{\dtprod}[2]{\ensuremath{\tb{#1} \cdot \tb{#2}}}

\newcommand{\coseno}[1]{\ensuremath{cos\,#1}}
\newcommand{\seno}[1]{\ensuremath{sin\,#1}}
\newcommand{\tang}[1]{\ensuremath{tan\,#1}}

\newcommand{\cosc}[1]{$cos^2\,#1$}
\newcommand{\senc}[1]{$sin^2\,#1$}
\newcommand{\tanc}[1]{$tan^2\,#1$}

\newcommand{\logb}[2]{\ensuremath{log_{#1}{#2}}}
\newcommand{\logn}[1]{\ensuremath{ln(#1)}}

\newtcolorbox{mybox}[2]{colback=#2!5!white,colframe=#2!75!black,fonttitle=\bfseries,title=#1, flushright title, width=\linewidth, height=5cm}

\newcommand{\price}{\ensuremath{\mathit{p} }}
\newcommand{\quantity}{\ensuremath{\mathit{q}} } 

\newcommand{\demandq}{\ensuremath{\mathit{q^D(p)}} }
\newcommand{\supplyq}{\ensuremath{\mathit{q^S(p)}} } 

\newcommand{\demandp}{\ensuremath{\mathit{p^D(q)}} }
\newcommand{\supplyp}{\ensuremath{\mathit{p^S(q)}} } 

\newcommand{\pprice}[1]{\ensuremath{\mathit{p}^{#1}}}
\newcommand{\pquantity}[1]{\ensuremath{\mathit{q}^{#1}} } 

\newcommand{\pdemandq}[1]{\ensuremath{\mathit{q^D(p^{#1})}} }
\newcommand{\psupplyq}[1]{\ensuremath{\mathit{q^S(p^{#1})}} } 

\newcommand{\pdemandp}[1]{\ensuremath{\mathit{p^D(q^{#1})}} }
\newcommand{\psupplyp}[1]{\ensuremath{\mathit{p^S(q^{#1})}} } 

\newcommand{\paxis}{\ensuremath{\mathit{p-axis}}} 
\newcommand{\qaxis}{\ensuremath{\mathit{q-axis}}} 

\newcommand{\lstar}[1]{\ensuremath{\mathit{{#1}^\ast}}}

\newcommand{\conj}[1]{\ensuremath{\mathcal{#1}}}
\newcommand{\estaen}[2]{\ensuremath{#1 \in \mathcal{#2}}}
\newcommand{\emptys}{\ensuremath{\emptyset}}

\newcommand{\bluebf}[1]{\textcolor{blue}{\textbf{#1}}}
\newcommand{\redbf}[1]{\textcolor{red}{\textbf{#1}}}

\renewcommand*{\thesection}{\arabic{section}}

\definecolor{myBlue}{HTML}{0088FF}

\titleformat{\section}[hang]{\Large\bfseries\sffamily}%
{\rlap{\color{myBlue}\rule[-6pt]{\textwidth}{1.2pt}}\colorbox{myBlue}{%
           \raisebox{0pt}[13pt][3pt]{ \makebox[60pt]{% height, width
                \fontfamily{phv}\selectfont\color{white}{\thesection}}
            }}}%
{15pt}%
{ \color{myBlue}#1
%
}
\titlespacing*{\section}{0pt}{3mm}{5mm}

\newcommand{\R}{%
\textcolor{blue}{\textit{R} }}
\newcommand{\python}{%
\textcolor{blue}{\textit{Python}}}
\newcommand{\pandas}{%
\textcolor{blue}{\textit{pandas}}}

\newcommand{\cunion}[2]{\ensuremath{#1 \cup #2}}
\newcommand{\cinter}[2]{\ensuremath{#1 \cap #2}}

\newcommand{\ccon}[1]{\ensuremath{\mathit{#1} }}

\newcommand{\func}[1]{\ensuremath{\mathit{f(#1)} }}
\newcommand{\funcc}[2]{\ensuremath{\mathit{#1(#2)} }}
\newcommand{\fder}[2]{\ensuremath{\mathit{{#1}'(#2)} }}
\newcommand{\fsder}[2]{\ensuremath{\mathit{{#1}"(#2)} }}
\newcommand{\fderd}[1]{\ensuremath{\frac{d\mathit{f}}{d\mathit{#1}}}}
\newcommand{\fnderd}[2]{\ensuremath{\frac{d\mathit{#1}}{d\mathit{#2}}}}
\newcommand{\fsderd}[1]{\ensuremath{\frac{d^2\mathit{f}}{d\mathit{#1}^2}}}
\newcommand{\eder}[1]{\ensuremath{\frac{d}{d\mathit{#1}}e^{#1}}}

\newcommand{\nexp}[2]{\ensuremath{#1^{#2}}}

\newcommand{\tend}[2]{\ensuremath{{#1}\,\rightarrow\,{#2}}}
\newcommand{\tfun}[1]{\ensuremath{\mathit{#1}}}

\newcommand{\parder}[2]{ \ensuremath{ \frac{\partial{#1}}{\partial{#2}} }}


\newcommand{\parderc}[2]{ \ensuremath{ \frac{\partial^2{#1}}{\partial{#2}} }}

\newcommand{\euros}[1]{ \ensuremath{#1 \text{€}} }

\makeatletter
%% The "\@seccntformat" command is an auxiliary command
%% (see pp. 26f. of 'The LaTeX Companion,' 2nd. ed.)
\def\@seccntformat#1{\@ifundefined{#1@cntformat}%
   {\csname the#1\endcsname\quad}  % default
   {\csname #1@cntformat\endcsname}% enable individual control
}
\let\oldappendix\appendix %% save current definition of \appendix
\renewcommand\appendix{%
    \oldappendix
    \newcommand{\section@cntformat}{\appendixname~\thesection\quad}
}
\makeatother

\definecolor{Mauve}{cmyk}{0,0.5,1,0}

\lstset{ %
  language=R,                     % the language of the code
  basicstyle=\tiny\ttfamily,     % the size of the fonts that are used for the code
  numbers=left,                   % where to put the line-numbers
  numberstyle=\tiny\color{gray},  % the style that is used for the line-numbers
  stepnumber=1,                   % the step between two line-numbers. If it's 1, each line
                                  % will be numbered
  numbersep=5pt,                  % how far the line-numbers are from the code
  backgroundcolor=\color{white},  % choose the background color. You must add \usepackage{color}
  showspaces=false,               % show spaces adding particular underscores
  showstringspaces=false,         % underline spaces within strings
  showtabs=false,                 % show tabs within strings adding particular underscores
  frame=single,                   % adds a frame around the code
  rulecolor=\color{gray},        % if not set, the frame-color may be changed on line-breaks within not-black text (e.g. commens (green here))
  tabsize=2,                      % sets default tabsize to 2 spaces
  captionpos=b,                   % sets the caption-position to bottom
  breaklines=true,                % sets automatic line breaking
  breakatwhitespace=false,        % sets if automatic breaks should only happen at whitespace
  title=\lstname,                 % show the filename of files included with \lstinputlisting;
                                  % also try caption instead of title
  keywordstyle=\color{blue},      % keyword style
  commentstyle=\color{DarkGreen}, % comment style
  stringstyle=\color{Mauve},      % string literal style
  escapeinside={\%*}{*)},         % if you want to add a comment within your code
  morekeywords={*,...}            % if you want to add more keywords to the set
} 